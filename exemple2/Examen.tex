% -*- coding: utf-8; -*-
\documentclass[a4paper,12pt]{article}

\usepackage[utf8]{inputenc}
\usepackage[T1]{fontenc}
\usepackage[french,british]{babel}


%%%%%%%%% Gestion de la correction

\usepackage{ifthen}
\usepackage{framed}
\usepackage{environ}
\usepackage{xcolor}
\colorlet{shadecolor}{gray!10!red!5}

\ifdefined\iscorrec     % Définition depuis le script d'appel
\else
  \def\iscorrec{1}      % 0=> sujet, 1=>corrigé
\fi

%%% \begin{correction}....\end{correction}
%   le contenu ne s'affiche que si \iscorrec=1

\NewEnviron{correction}{\ifthenelse{\iscorrec=1}{%
    \medskip\medskip\par\begin{shaded}\noindent{\bf Solution:} \BODY \end{shaded}
}{}}


%%%%%%%%% Paramètres individuels du sujet

%%% Valeurs par défaut (pour une compilation isolée)

\ifdefined\assignedparameters\else                           
    \def\assignedparameters{1,1,Toto}   % dans cet exemple, 3 paramètres requis
\fi

%%% Liste permettant d'accéder aux assignedparameters: \ListAP[1], \ListAP[2], etc.

\usepackage{listofitems}
\ifx\assignedparameters\empty       % (erreur de \readlist dans ce cas)
\else
    \readlist*\ListAP{\assignedparameters}
\fi

%%% Switch de versions, définie ici "à vide" pour mettre du renewcommand plus tard

\newcommand{\switch}{}


%%%%%%%%%%%%%%%%%%%%%%%%%%%%%%%%%%%%%%%%%%%%%%%%%%%%%%%%%%%%%%%%%
%%%%%%%%%%%%%%%%%%%%%%%%%%%%%%%%%%%%%%%%%%%%%%%%%%%%%%%%%%%%%%%%%
%%%%%%%%%%%%%%%%%%%%%%%%%%%%%%%%%%%%%%%%%%%%%%%%%%%%%%%%%%%%%%%%%
%%%%%%%%%%%%%%%%%%%%%%%%%%%%%%%%%%%%%%%%%%%%%%%%%%%%%%%%%%%%%%%%%

\begin{document}

\noindent BUT, 1ère année \hfill 2021-22

\begin{center} \Large Examen de math \end{center}


%%%%%%%%%%%%%%

\subsection*{Exercice 1}

% Dans cet exemple, 1er paramètre assigné = version de l'exo 1 (de 1 à 3). 
% On nomme la variable en question (pas nécessaire mais plus clair):
\def\versionExo{\ListAP[1]} 

\ifnum\versionExo=1
    Calculer $a=3+7$.
\fi
\ifnum\versionExo=2
    Calculer $a=3\times 7$. 
\fi
\ifnum\versionExo=3
    Calculer $a=21-12$.
\fi
Le nombre $a$ est-il pair ou impair~?

% S'il y a souvent besoin d'effectuer cet affichage différencié, on peut inclure la mécanique de différenciation dans une macro d'usage simplifié :
%       \switch{v1}{v2}{v3}
\renewcommand{\switch}[3]{%
    \ifnum\versionExo=1 #1\fi
    \ifnum\versionExo=2 #2\fi
    \ifnum\versionExo=3 #3\fi
}

% ce qui donne comme usage (ici dans la correction):
\begin{correction}
    $a=\switch{10}{21}{9}$. Ce nombre est \switch{pair}{impair}{impair}.
\end{correction}



%%%%%%%%%%%%%%

% Dans cet exemple, 2e paramètre assigné = version de l'exo 2 (de 1 à 4).
% Dans cet exemple, 3e paramètre assigné = prénom du protagoniste (ok ça sert à rien^^).
% On nomme les variables en question (pas nécessaire mais plus clair):
\def\versionExo{\ListAP[2]} 
\def\prenom{\ListAP[3]} 

% Pour se simplifier la vie
\renewcommand{\switch}[4]{%
    \ifnum\versionExo=1 #1\fi
    \ifnum\versionExo=2 #2\fi
    \ifnum\versionExo=3 #3\fi
    \ifnum\versionExo=4 #4\fi
}

\subsection*{Exercice 2}

Soit le nombre
\[
b=\switch{1000}{10}{1}{0.01}
\]
Aidez \prenom\ à exprimer $b$ comme une puissance de 10.

\begin{correction}
On a $b=10^{\switch{3}{1}{0}{-2}}$.
\end{correction}


\end{document}


